\chapter{Conclusions}
\label{chapter: conclusions}

\section{On the difficulty of this lab assignment}

\paragraph{}
We were surprised about the time we had to spend in Part 1, especially knowing
this part is arguably the easiest of the two requested for this lab assignment.
Firstly, we had a tough time thinking how to model the problem as a SAT problem;
we were blinded by the idea of assigning car positions in the parking, while the
problem was to simply check if the (already assigned) cars were not blocked.
Secondly, we spent more time than we would have wanted because of implementation
details; at some point we experimented the need to perform a good refactoring to
our code.

\paragraph{}
With Part 2, the implementation was pretty straightforward, and Python3's simple
syntax was of great help for us to have our program up and running in no time.
However, no matter how well did the program work, we know that the difficulty of
this second part relies on how do we design states, operators and the heuristic
function.

Anyway, our code to solve the second part of the assignment is not fully
working. We have encountered some problems on how Python manages objects. Our
code works for very simple configurations that only need to expand the initial
node to reach the solution. Although no solution is returned on most of the
cases, we think that our approach is correct, as I/O files, expanding of nodes,
calculation of costs and heuristics work.

\section{On the knowledge we have acquired and/or reinforced}

\paragraph{}
This lab assignment has made us look at SAT from a different point of view. We have found Part 2 of the lab assignment quite useful, since modelling the requested problem as a search problem is a skill we still have to acquire by practice. This lab assignment has served as a perfect search exercise with regard to the final exam we have in January, where we will have for sure an exercise on search.
