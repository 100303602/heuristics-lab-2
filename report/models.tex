\chapter{Models Description}
\label{chapter: models description}








%----------------------------------------- PART 1 --------------------------------------------------

\section{Part 1: SAT verification of the parking configuration}

\paragraph{}
For this part we had to refrain ourselves from the temptation of modelling the problem as an assignment of cars to park positions, since these positions were already fixed. Instead, we have to somehow \textit{check} that an (already present) assignment was correct. In other words, the configuration read from the input file should not be changed; JaCoP should not decide which values to assign to the boolean variables that describe the parking. We have to make these values \textit{fixed}, expressing this with propositional logic. One way we thought to solve this problem is to include this information as disjunctions of only one literal, and adding each one them to the CNF with a conjunction.

\subsection{Variables}

\paragraph{}
Variables have been defined as follows:
\begin{itemize}
  \item $c_{ij}$: there is a car at lane $i$ and position $j$
  \item $b^{d)}_{ij}$: car adjacent to that at $i,j$ entered the lane before
  \item $s^{d)}_{ij}$: car adjacent to that at $i,j$ has a category with the same waiting time 
  \item $l^{d)}_{ij}$: car adjacent to that at $i,j$ has a category with lower waiting time
\end{itemize}
Where $d \in {L,R}$. $L$ represents information about the adjacent car at the left of car at $i,j$, and $R$ represents information about the adjacent car at the right of car at $i,j$.

\subsection{Propositional formula}

\paragraph{}
The formula can be divided in two parts: the part that checks for blocked cars, and the part that represents the parking configuration by using 1-variable clauses. It should be noticed that literals in the formula may be non-negated ($b^L_{ij}$) or negated ($\neg b^d_{ij}$), depending on the input file configuration. Since it would be incorrect to include a single formula (as literals on the right are non-negated or negated depending on the input file), the following equation represents a group of formulas, hence the use of set notation for the right part.

\paragraph{}
Therefore we get the following propositional formula:
\begin{equation}
  \bigand{\substack{i \in N,\\ j \in (0, N-1)}}l^{L)}_{ij} \lor (s^{L)}_{ij} \land b^{L)}_{ij}) \lor l^{R)}_{ij} \lor l^{R)}_{ij} \lor (s^{R)}_{ij} \lor b^{R)}_{ij})
  \bigand{\substack{
      \alpha \in\\
      B^{L)} \cup
      B^{R)} \cup \\
      S^{L)} \cup
      S^{R)} \cup \\
      L^{L)} \cup
      L^{R)} \cup \\
      C
    }}
    \alpha
\end{equation}
where
\begin{itemize}
  \item $B^{L)}$: matrix containing literals $b^{L)}_{ij}$ or $\neg b^{L)}_{ij}$ as loaded from the input file
  \item $S^{L)}$: matrix containing literals $s^{L)}_{ij}$ or $\neg s^{L)}_{ij}$ as loaded from the input file
  \item $S^{R)}$: matrix containing literals $s^{R)}_{ij}$ or $\neg s^{R)}_{ij}$ as loaded from the input file
  \item $L^{L)}$: matrix containing literals $l^{L)}_{ij}$ or $\neg l^{L)}_{ij}$ as loaded from the input file
  \item $L^{R)}$: matrix containing literals $l^{R)}_{ij}$ or $\neg l^{R)}_{ij}$ as loaded from the input file
  \item $C$: matrix containing literals $c_{ij}$ or $\neg c_{ij}$ as loaded from the input file
\end{itemize}

\paragraph{}
Turning this formula into CNF we get the following:
\begin{equation}
  \bigand{\substack{i \in N,\\ j \in (0, N-1)}}
  (l^{L)}_{ij} \lor l^{R)}_{ij} \lor s^{L)}_{ij} \lor s^{R)}_{ij}) \land
  (l^{L)}_{ij} \lor l^{R)}_{ij} \lor s^{L)}_{ij} \lor b^{R)}_{ij}) \land
  (l^{L)}_{ij} \lor l^{R)}_{ij} \lor s^{R)}_{ij} \lor b^{L)}_{ij}) \land
  (l^{L)}_{ij} \lor l^{R)}_{ij} \lor b^{L)}_{ij} \lor b^{R)}_{ij})
  \bigand{\substack{
      \alpha \in\\
      B^{L)} \cup
      B^{R)} \cup \\
      S^{L)} \cup
      S^{R)} \cup \\
      L^{L)} \cup
      L^{R)} \cup \\
      C
    }}
    \alpha
\end{equation}

\paragraph{}
Notice that the formula does not check literals related to the sides of the parking lot, since these are never going to be blocked.

%----------------------------------------- PART 2 --------------------------------------------------

\section{Part 2: Heuristic Search}

\subsection{States}

\paragraph{}
\warning{Guille}{Completar}

\subsection{Operators}

\paragraph{}
\warning{Guille}{Completar}

\subsection{Initial state(s)}

\paragraph{}
\warning{Guille}{Completar}

\subsection{Final state(s)}

\paragraph{}
\warning{Guille}{Completar}

\subsection{Heuristic function}

\paragraph{}
\warning{Guille}{Completar}
